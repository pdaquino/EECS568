\title{Answers to Problem Set 1}
\author{
				Lauren Hinkle\\
				Robert Goeddel \\
        Pedro Barbuda \\
        Yash Chitalia}
\date{\today}

\documentclass[12pt]{article}
\usepackage{graphicx}
\usepackage{amsmath}

\begin{document}
\maketitle

\section{Linear Algebra Review}

\paragraph{A}

\paragraph{B}
\begin{equation}
A = \left( \begin{array}{ccc} 1 & 0 & 0 \\ 0 & 1 & 0 \end{array} \right)
\end{equation}

\paragraph{C}
The idea is that $AA^T = I \Rightarrow A^TA=I$ only if $A$ is square. We proved it by showing that if $A$ is rectangular it only works one way around (i.e. if $AA^T=1$ then $A^TA \neq I$).

\paragraph{D}

\section{Multiavariate Gaussians}

\paragraph{A}
Let $N$ be the number of samples, $K$ be the number of random variables in $\mathbf{x}$ and $\mathbf{\mu}$ be the sample mean vector. The $i$-th sample of the random variable $\mathbf{x}_k$ is denoted by $\mathbf{x}_{ik}$ Then each element of $\mathbf{\mu}$ is going to be of the form:

\begin{equation}
\mu_k = \frac{1}{N}\displaystyle\sum_{i=1}^{N}{x_{ik}}, k=1,2,...,K
\end{equation}

And every element $\sigma_{jk}$ of the $K \times K$ covariance matrix $\Sigma$ will be given by:

\begin{equation}
\sigma_{jk}=\frac{1}{N-1}\displaystyle\sum_{i=1}^{N}{(x_{ij}-\mu_j)(x_{ik}-\mu_{k})}
\end{equation}


\end{document}
