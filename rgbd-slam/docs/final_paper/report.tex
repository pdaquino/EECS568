\documentclass[letterpaper, 10pt, conference]{ieeeconf}

\IEEEoverridecommandlockouts    % If we want to use the thanks command

\usepackage{color}
\usepackage{graphicx}
\usepackage{amsmath}
\usepackage{amssymb}

% EPS2PDF stuff?

\newcommand{\xxx}[1]{\textcolor{red}{#1}}

\title{\LARGE \bf
    3D Environment Reconstruction Using RGB-D Visual Odometry
}

\author{Pedro D'Aquino, Rob Goeddel, Lauren Hinkle, and John Peterson}


\begin{document}

\maketitle
\thispagestyle{empty}
\pagestyle{empty}

\begin{abstract}
The reconstruction of environments in 3D can be challenging with just range
data. RGB-D sensors like the Microsoft Kinect provide visual details that
allow odometry to be used even in environments with ambiguous depth data.
This paper demonstrates a simplified implementation of Henry et al.'s work
on RGB-D Mapping.~\cite{Henry10rgbd}
Our system can construct reasonably accurate models of
small to medium sized indoor environments in real time at resolutions of up
to 2\,cm and in near-real time at 1\,cm. By using SURF features from the image
data in conjunction with the Kinect's depth data, we are able to compute
accurate alignments of point clouds over time. Our system does not attempt to
implement the loop-closing portion of Henry et al.'s work.
\end{abstract}

\section{Introduction}


\section{Motivation}
\xxx{BACKGROUND AND MOTIVATION GOES HERE}

\section{Methodology}
\xxx{METHODOLOGY GOES HERE}

\section{Evaluation}
\xxx{EVALUATION AND PRETTY FIGURES GO HERE}

\section{Conclusion}
\xxx{CONCLUSION GOES HERE}

% XXX References
\bibliographystyle{IEEEtran}
\bibliography{sources}

\end{document}
